\documentclass{article}

\usepackage{mathtools}

\DeclarePairedDelimiter\abs{\lvert}{\rvert}%
\DeclarePairedDelimiter\norm{\lVert}{\rVert}%
\makeatletter
\let\oldabs\abs
\def\abs{\@ifstar{\oldabs}{\oldabs*}}

\usepackage[utf8]{inputenc}
\usepackage{pgfplots}

\pgfplotsset{
    discard if not/.style 2 args={
        x filter/.code={
            \edef\tempa{\thisrow{#1}}
            \edef\tempb{#2}
            \ifx\tempa\tempb
            \else
                \def\pgfmathresult{inf}
            \fi
        }
    }
}

\usepackage{subcaption}
\usepackage{gensymb}
\usepackage{amsmath,amsfonts,amssymb,amsthm,epsfig,epstopdf,titling,url,array}
\usepackage{enumerate}
\usepackage[a4paper, total={6in, 8in}]{geometry}

\usepackage{algorithm}
\usepackage[noend]{algpseudocode}

\makeatletter
\def\BState{\State\hskip-\ALG@thistlm}
\makeatother

\newcommand{\mathbbm}[1]{\text{\usefont{U}{bbm}{m}{n}#1}}
\newtheorem{thm}{Theorem}[section]
\newtheorem*{thmt*}{Theorem}
\newtheorem{lem}[thm]{Lemma}
\newtheorem{assumption}{Assumption}
\newtheorem{prop}[thm]{Proposition}
\pgfplotsset{compat=1.15}

\newtheorem{defn}{Definition}[section]
\title{Auditing for Multi-Differential Fairness of Black Box Classifiers}
\author{}
\date{January 2019}

\begin{document}
\maketitle
\section{Introduction}
Machine learning algorithms are more and more used to make support decisions that have adverse consequences on an individual's life: for example, classifiers have supported the judicial system to decide whether a criminal offender is likely to recommit a crime; or lender to determine the default risk of a potential borrower. At issue is whether classifiers are fair in the sense of \cite{calsamiglia2009decentralizing} , that is whether classifiers' outcomes are independent of \textit{exogenous irrelevant characteristics} (\cite{calsamiglia2009decentralizing}) or protected attributes, including race and gender. Abundant examples of classifiers' discrimination can be found in many applications (see \cite{NY2017}, \cite{atlantic2016}, \cite{ProPublica2016}). A ProPublica story (\cite{ProPublica2016}) reported that a machine learning based risk assessment tool, COMPASS, assigns higher risk to Afro-American defendants than they actually have; and lower risk to Caucasian defendants.

\bigskip
Contestability is challenging for potential victims of machine learning discrimination because (i) many assessment tools are proprietary and are not always required to be transparent about their functioning; and, (ii) there is, at least in the United States, a paper trail of legal cases that have put the burden on the plaintiff to demonstrate disparate treatment, that is to establish that characteristics irrelevant to the task affect the algorithm's outcomes (e.g. see \textit{Ricci et al. vs DeStefano et al.}\cite{Ricci}, \textit{Loomis vs. the State of Wisconsin} \cite{Loomis}). In  \textit{Loomis vs. the State of Wisconsin} (\cite{Loomis}) , the Wisconsin Supreme Court rules in favor of the use of COMPASS in recidivism risk assessment because, other among things, the plaintiff "failed to meet his burden of showing that the sentencing court actually relied on gender as a factor in sentencing". 
	
\bigskip
 This paper defines a framework -- differential fairness -- and provides a tool for individuals to contest the outcomes of black boxes classifiers whose design, inputs and validation procedures are unknown. Differential fairness is a guarantee that a classifier's outcome are nearly independent of protected attributes conditional on features relevant to the decision making process. The term "differential" is used to emphasize that in our framework a classifier is fair if its outcomes are statistically nearly identical for two individuals that differ only by their protected attribute. The concept borrows from the differential privacy literature (see \cite{dwork2014algorithmic}): from a privacy perspective, differential fairness  measures how much information is leaked by the classifier's outcomes on the distribution of protected attributes. Compared to previous definitions of fairness based on information leakages (e.g. \cite{feldman2015certifying}) is that the leakage is measured conditional on an individual's features. Other individual-level notion of fairness have been proposed (see \cite{dwork2012fairness}) have been proposed, but they rely on similarity metric that has been proved difficult to define in practice. Differential fairness me Therefore, the model allows to measure anThe main advantage of differential fairness is that by conditioning on the individual's features, the framework allows to  measure the causal effect of protected attributes on the classifier's outcomes. 
 
 \bigskip
 However, because it would require to search through a prohibitively large space of individuals, differential fairness cannot be audited for at the individual level. To achieve our goal of auditing for disparate treatment, we first relax the definition of differential fairness to a notion of multi-differential fairness that guarantees that the classifier's outcomes do not leak any information about the distribution of protected attributes for all sub-populations in a rich collection of groups of individuals. 
 
 \bigskip
 The relaxation to multiple individuals definition of fairness has allowed recent contributions, including \cite{kim2018fairness}, \cite{hebert2017calibration} or \cite{kearns2017preventing}, to efficiently audit for fairness at a small granularity. In this paper,  we show that the relaxation to multi-differential fairness allows to efficiently obtain three fairness diagnostics.  First, we reduce certifying the lack of multi-differential fairness into a learning problem to predict for which individuals protected attribute coincides with the classifier's outcome. Secondly, we propose an algorithm to efficiently find the sub-population for which the classifier violates the most differential fairness. Lastly, for any individual we define disparate treatment as the most harm made by the classifier to a subpopulation the individual belongs to. Consider, for example, Bob, Afro-American, who is assessed for recidivism risk. Our tool allows to measure whether Bob belongs to a sub-population for which the probability to be assessed a high recidivism risk is twice as large for Afro-American than for other races. This fairness diagnostic is essential information for an individual to be able to contest a classifier's outcome in court. More generally our auditing tool aims at reinforcing individual ability to recourse a harmful decisions supporter by a machine learning algorithm (see also \cite{ustun2018actionable}, \cite{russell2019efficient} for recent efforts on contestability and recourse).  
 
 \paragraph{Related Work}
 
 
 \paragraph{Contributions}
Our contributions are as follows:
\begin{itemize}
	\item We introduce a concept of differential fairness to measure how a classifier leaks information related an individual's protected attribute. 
	\item With a relaxed definition of differential fairness, we reduce the problem of certifying for the lack of differential fairness to the problem of predicting when a binary attribute contribute coincides with the classifier's outcome. This reduction allows to efficiently assess whether the exists a sub-population for which the classifier leaks information related to the distribution of its protected attribute. We show correctness and sample complexity for our certifying algorithm.
	\item We develop an algorithm, worst violation,  to efficiently identify the  sub-population that is the most harmed by a potential classifier's discrimination. This is essential not only to audit the mechanisms underlying the classifier's outcomes, but also to offer recourse to the individuals identified as the most harmed. 
	\item We test the validity of our algorithms, certifying and worst violation, with synthetic datasets.
	\item We apply our tools on a recidivism risk assessment in Broward County, Florida. We certify that the existence of a statistically significant level of differential unfairness for Afro-American individuals. We do not find the same level of significance for male. Moreover, we identify the characteristics of the sub-population for which the risk assessment appears to discriminate the most severely against Afro-American. 
\end{itemize}
 
 
  



\section{Individual and Multi-Differential Fairness}

\subsection{Preliminary}

\paragraph{Notations}
An individual $i$ is defined by a tuple $((x_{i}, a_{i}), y_{i})$, where $x_{i}\in \mathfrak{X}$ denotes individual $i$'s audited features; $a_{i}\in\mathfrak{A}$ denotes her protected attribute; and $y_{i}\in \{-1, 1\}$ is the classification provided by a black box classifier $f$. The auditor draw samples $
\{((x_{i}, a_{i}), y_{i})\}_{i=1}^{m}$ of size $m$ from a distribution $D$ on $\mathfrak{X} \times \mathfrak{A}\times \{-1, 1\}$. 

\bigskip
Features in $\mathfrak{X}$ are not necessarily the ones used to train $f$. First, the auditor may not have access to all features used to train $f$. Secondly, the auditor may decide to leave deliberately some features used to train $f$ out of $\mathfrak{X}$ because she believes that those features should not be used to define similarity among individuals. For example, if $f$ classifies loan according to their probability of repayment, the auditor may consider that credit score should be used to define individual similarity, but that zipcode, because correlated with races, should not be an auditing feature, although it was used to train $f$.  

\paragraph{Assumptions}
In our analysis we make the following assumption:
\begin{assumption}
\label{ass: 1}
For all $x\in \mathfrak{X}$, $Pr[A|X=x] > 0.$
\end{assumption}
Assumption \ref{ass: 1} guarantees that the distribution of auditing features conditional on protected attributes have common support: there is no $x\in \mathfrak{X}$ that reveals perfectly the individual's protected attribute.  

\subsection{Individual Differential Fairness}
\paragraph{Individual Differential Fairness} 
We define differential fairness as the guarantee that conditional on features relevant to the tasks, a classifier's outcome is nearly independent of protected attributes: 

\begin{defn}(Individual Differential Fairness)
\label{def: idf}
For $\delta \in [0, 1)$, a classifier $f$ is $\ \delta-$ differential fair if for all $x\in\mathfrak{X}$ and all $a\in \mathfrak{A}$ and for all $y\in\{-1, 1\}$
\begin{equation}
\label{eq: idf}
   e^{-\delta} \leq \frac{Pr[Y=y|A =a, x]}{Pr[Y=y|A\neq a, x]} \leq e^{\delta}
\end{equation}
\end{defn}

The parameter $\delta$ controls how much the distribution of the classifier's outcome $Y$ depends on protected attributes $A$ conditional on auditing features $x$: larger value of $\delta$ implies larger leakage. For example, for a classifier that does not satisfy the fairness condition \ref{def: idf} with $\delta=\ln(2)$, there exist individuals $x$ that are twice as likely to be classified $y=1$ if $A=1$ than if $A=-1$. 

\paragraph{Differential Fairness and Distance of Distributions}
Since the space of auditing features $\mathfrak{X}$ does not necessarily correspond to the one used to trained the classifier $f$, for any $x\in \mathfrak{X}$, the auditor access samples drawn from two distributions $Y|A=a, x$ and $Y|a\neq a, x$. Individual fairness constraints these two distributions too be nearly identical when distribution similarity is defined by max-divergence. Formally, the max divergence of two distributions $P$ and $Q$ is defined as 

\begin{equation}
D_{\infty}(P||Q) = \max_{y\in Y}\ln\left(\frac{Pr[P=y]}{Pr[Q=y]}\right)
\end{equation}
The fairness condition in \ref{def: idf} can be equivalently rewritten in terms of max divergence as:
\begin{equation}
\label{eq: maxdiv_idf}
D_{\infty}((A|x, Y)||(A| x)) \leq \delta
\end{equation}
If a classifier is $\delta$-individual fair, the posterior and prior distribution of protected attribute conditional on auditing features is bounded by $\delta$. 

\paragraph{Relation with Differential Privacy}
There is an analogy between individual differential fairness for classifiers and differential privacy for database queries. Differential privacy as in  \cite{dwork2014algorithmic} guarantees that outcomes from a query are not distinguishable when computed on two adjacent databases that differs only by one record. The fairness condition \eqref{eq: idf} implies that outcomes from a classifier are not distinguishable for individuals that differ only by their protected attributes. The max-divergence equivalence in \ref{eq: maxdiv_idf} shows that differential fairness bounds the information leakage caused by $Y$ conditional on what is already leaked by the auditing features $x$ [placeholder: why does the analogy matter? Possibly, (i) merges the field of fairness with privacy, a field where computer science is "more comfortable" with; (ii). Why is there no balance issue in differential privacy? ] 


\paragraph{Individual Fairness}
The definition \ref{def: idf} is an individual level definition of fairness, since it conditions the information leakage on auditing features $x$. Compared to the notion of individual fairness in \cite{dwork2012fairness}, individual differential fairness does not require to explicit a similarity metric. This is important because defining a similarity metric has been the main limitation of applying the concept of individual fairness. The similarity of treatment in differential fairness is defined in a statistical sense as the max-divergence distance between the distributions $Y|(x, A=a)$ and $Y|(x, A\neq a)$. Differential fairness interprets disparate treatment of a classifier $f$ on an individual with auditing features $x$ as a non-negligible distance between $Y|(x, A=a)$ and $Y|(x, A\neq a)$. 

\paragraph{Intention in Disparate Treatment}
Differential fairness is a useful definition of fairness in machine learning because it provides a test to whether the protected attribute $A$ affects in a causal sense the classifier's outcome. This is important because, at least in the United States, there are legal precedents (see \cite{Ricci}, \cite{Loomis}, Title VII) that require a plaintiff to demonstrate the disparate treatment was intentional. Causality is defined here as the existence of path (either direct or indirect) between the protected attribute and the classifier's outcome that is not blocked by the auditing features $x$. Therefore, differential fairness sets a framework to establish causation of protected attributes on a classifier's outcome conditional on the auditing feature space. 

\subsection{Multi-differential fairness}
Although useful, the notion of individual differential fairness suffers from one limitation: it cannot be computationally efficiently audited for. Looking for violations of individual differential fairness will require searching over a set of $2^{|\mathfrak{X}|}$ individuals. Moreover, if $\mathfrak{X}$ is rich enough empirically, a sample from a distribution over $\mathfrak{X} \times \mathfrak{A}\times \{-1, 1\}$ has a negligible probability to have two individuals with the same auditing feature $x$ and different protected attributes $a$. 

\bigskip
Therefore, we relax the definition of individual differential fairness and impose differential individual fairness for group of individuals or sub-populations. Formally, $\mathfrak{C}$ denotes a collection of subsets $S$ of $\mathfrak{X}$. The collection $\mathfrak{C}_{\alpha}$ is $\alpha$-strong if for $S\in \mathfrak{C}$ and $y\in \{-1, 1\}$, $Pr[Y=y \;\&\; x\in S] \geq \alpha$.  

\begin{defn}(Multi-Differential Fairness)
\label{def: mdf}
Consider a $\alpha$-strong collection $\mathbb{C}_{\alpha}$ of sub-populations of $\mathfrak{X}$. For $0\leq \delta$, a classifier $f$ is $(\mathbb{C}_{\alpha}, \delta)$-multi differential fair with respect to $\mathfrak{A}$ if for all protected attributes $a, a^{'}\in \mathfrak{A}$, $y\in\{-1, 1\}$ and for all $S\in \mathbb{C}_{\alpha}$:
\begin{equation}
\label{eq: mdf}
Pr[Y=y|A=a, S] \leq e^{\delta} Pr[Y=y|A=a^{'}, S]
\end{equation}
\end{defn}

Multi-differential fairness relaxes the notion of differential fairness by protecting sub-populations instead of individuals. Multi-differential fairness guarantees that the outcome of a classifier $f$ is nearly mean-independent of protected attributes within any sub-population $S\in \mathfrak{C}_{\alpha}$. The parameter $\delta$ controls for the amount of information related to protected attributes that the classifier leaks: smaller value of $\delta$ means smaller leakage. The fairness condition \ref{eq: mdf} applies only to subpopulations with $Pr[Y=y \;\&\; x\in S] \geq \alpha$ for $y\in\{-1, 1\}$. This is to avoid trivial cases where $\{x\in S \; \& \; Y=y\}$ is a singleton for some $y$, which would imply that $\delta=\infty$. 

\paragraph{Disparate Treatment versus Disparate Impact}
It is interesting to compare the definition of multi-differential fairness in \ref{def: mdf} with previous definitions of fairness that are based on information leaked by the data about the protected attributes.  First, if auditing features is empty, then multi-differential fairness devolves into a notion of disparate impact as in \cite{feldman2015certifying} or in \cite{chouldechova2017fair}. Disparate impact is then a particular case of multi-differential fairness where no disparate treatment  statement is made. On the other hand, if auditing features are all the features but protected attributes used to train $f$, multi-differential fairness is a framework to test whether there exists a sub-population for which there is a direct causal effect of protected attributes on the classifier's outcomes. 


\paragraph{Collection of Indicators.}
The collection of sub-population $\mathbb{C}$ can be equivalently thought as a family of indicators: for each $S\in \mathbb{C}$, there is an indicator $c_{S}: \mathfrak{X}\rightarrow \{-1, 1\}$ such that $c_{S}(x)=1$ if and only if $x\in S$. The relaxation of differential fairness to a collection of groups or sub-population is akin to \cite{kim2018fairness}, \cite{kearns2017preventing} or \cite{hebert2017calibration} where $\mathfrak{C}$ is the computational bound on the granularity of their definition of fairness. The richer $\mathbb{C}$, the stronger the fairness guarantee offers by definition \ref{def: mdf}. However, the complexity of $\mathbb{C}$ is limited by the fact that we identify a sub-population $S$ via random samples drawn from a distribution over $\mathfrak{X} \times \mathfrak{A}\times \{-1, 1\}$. The rest of this paper shows that auditing for multi-differential fairness in polynomial time requires to limit the complexity of $\mathbb{C}$. Potential candidates for $\mathbb{C}$ will be the family short-decision trees or the set of conjunctions of constant number of boolean features. Therefore, auditing for multi-differential fairness will not check whether the fairness condition \eqref{eq: mdf} holds for \textit{all} sub-populations of $\mathfrak{X}$, but only check  the fairness condition for all sub-populations that can be \textit{efficiently identifiable}. 


\section{Fairness Diagnostics}

The definition of multi-differential fairness requires to verify that in no sub-population $S\in \mathbb{C}$ with $Pr[S]\geq \alpha$, the classifier leaks information about the distribution of protected attributes. If $\mathbb{C}$ is a rich and large class of subsets of the feature space $\mathfrak{X}$, an auditing algorithm linearly dependent on $|\mathbb{C}|$ can be prohibitively expensive. In this section we show that finding an auditing algorithm reduces to agnostic learning of the class of sub-populations $\mathbb{C}$. That is, there is no $\log(\mathbb{C})$ running time auditing algorithm unless $\mathbb{C}$ is efficiently agnostically learnable.  

\subsection{Certifying (the Lack) Fairness and Agnostic Learning}
Auditing for multi-differential fairness consists firstly, in establishing there exists a fairness violation; secondly, in identifying a sub-population $S$ that violates the most the fairness condition in \ref{def: mdf}. 

\paragraph{Multi Differential Fairness and Balanced Distribution}
The fairness condition \ref{def: mdf} is unchanged if the feature distribution is reweighted, as long as the reweighting scheme does not depend on the classifier's outcome $Y$. More formally, for any weights $u: \mathfrak{X}\times \mathfrak{A} \rightarrow \mathbb{R}$ such that $u(x,a)> 0$ and $E[u]=1$, 

\begin{equation}
Pr[Y|A=a, S] \leq e^{\delta} Pr[Y|A=a^{'}, S] \iff Pr_{u}[Y|A=a, S] \leq e^{\delta} Pr_{u}[Y|A=a^{'}, S],
\end{equation}
the sub-script $u$ indicating that the probability are taken over the reweighted distribution. 

\bigskip
Suppose that for any $a \in \mathfrak{A}$, we have oracle access to the importance sampling weight $w_{a}(x)=\frac{1 - P[A=a|x]}{P[A=a|x]}$. For any distribution $D_{f}$ over $\mathfrak{X} \times \mathfrak{A}\times \{-1, 1\}$ denote $D_{f}^{w}$ the corresponding balanced distribution. Note that once reweighted by $w_{a}$, for any sub-population $S\in \mathbb{C}$, auditing features does not reveal anything about the distribution of the protected attribute $A$: $Pr_{w}[A=a|X, S]=Pr_{w}[A\neq a|X, S]$. With a balanced distribution, the multi differential fairness condition can be rewritten as follows: for all protected attributes $a\in \mathfrak{A}$, $y\in \{-1,1\}$ and for all $S\in \mathbb{C}_{\alpha}$

\begin{equation}
    \label{eq: mdf_w}
    Pr_{w}[A=a |S, y] \leq \frac{e^{\delta}}{e^{\delta} + 1},
\end{equation}
where the sub-script $w$ reminds that the distribution $D_{f}^{w}$ is balanced. Since the distribution $D_{f}^{w}$ induced by $f$ is balanced, auditing features $x$ do not reveal any information on protected attributes and multi-differential fairness can then be interpreted as an upper bound on ability to predict $A$ given the classifier's outcome for any sub-population $S\in \mathbb{C}$ with $Pr_{w}[S, y] \geq \alpha$ for $y\in\{-1, 1\}$.  A violation of $(\mathbb{C}_{\alpha}, \delta)$- multi differential fairness is a sub-population $S\in \mathbb{C}_{\alpha}$ such that 

\begin{equation}
    \label{eq: mdf_viol1}
    Pr_{w}[A=a |S, y] - \frac{1}{2}\geq \frac{e^{\delta}}{e^{\delta} + 1} - \frac{1}{2}. 
\end{equation}

Thereofre, a $\gamma$-unfairness certificate is a subset $S\in \mathbb{C}$ such that there exists $y\in \{-1, 1\}$ with
\begin{equation}
    Pr_{w}[S, y]\left\{Pr_{w}[A=a |S, y] -\frac{1}{2}\right\}\geq \gamma,
\end{equation}
with $\gamma =\alpha \left(e^{\delta}/(1+e^{\delta})-1/2\right)$. $\gamma$ is then a measure of multi-differential unfairness that combines the size of the sub-population where a violation exists and the magnitude of the violation. With balanced distribution, certifying multi-differential fairness is akin to searching for $\gamma$-unfairness certificate. 

\begin{defn}(Certifying Multi-Differential Fairness). 
Let $\gamma, \epsilon >0$, $\eta\in (0, 1)$ and $\mathbb{C}_{\alpha}$ be an $\alpha$-strong collection of sub-populations in $\mathfrak{X}$. An $(\epsilon, \eta)$- certifying algorithm $M(\epsilon, \delta)$ is an algorithm that for any sample from a distribution $D_{f}$ induced by a classifier over $\mathfrak{X} \times \mathfrak{A}\times \{-1, 1\}$, outputs a $\gamma-\epsilon$-unfairness certificate with probability $1-\eta$ whenever $f$ is $ \gamma$-multi differential unfair; and, certifies fairness with probability $1-\eta$ whenever $f$ is is $\gamma$-multi differential fair. 

\bigskip
Moreover, $M(\epsilon, \delta)$ is an efficient certifying algorithm if it requires $poly(\log(|\mathfrak{C}_{\alpha}|, \log(1/\eta), 1/\epsilon))$ samples and runs in $poly(\log(|\mathfrak{C}_{\alpha}|, \log(1/\eta), 1/\epsilon))$. 
\end{defn}

Searching for $\gamma$-unfairness certificate can be formulated as a problem of detecting correlations that can be written as:

\begin{equation}
Pr_{w}[AY=c_{S}] \geq 1 - \rho(y) + 4\gamma
\end{equation}
with $\rho(y)=Pr_{w}[A=Y]$. Therefore, certifying the lack of multi-differential fairness can be phrased as a weak agnostic learning problem. A concept class $\mathbb{C}$ is agnostically efficiently learnable if and only if for all $\epsilon, \eta >0$, there exists an algorithm $\mathfrak{M}$ that given access to a distribution $\{x_{i}, o_{i}\}\sim D\times \{-1, 1\}$ outputs with probability $1-\eta$ in $poly(\log(|\mathfrak{C}, \log(\frac{1}{\eta}), \epsilon)$ outputs  a function $h\in \mathbb{C}$ such that
$$ Pr_{D}[g=h]  + \epsilon \geq max_{c\in \mathbb{C}}Pr_{D}[g=c]. $$
We show that if the collection of subpopulation $\mathbb{C}$ admits an efficient agnostic learner, we could use that learner to construct an algorithm certifying multi-differential fairness.

\begin{thm}
\label{thm: al}
Let $\epsilon > 0$, $\beta >0$ and $\mathbb{C}\subset 2^{\mathfrak{X}}$. There exists an efficient $(\epsilon, \eta)$-auditing algorithm for $\mathbb{C}$ on balanced distributions if and only if $\mathbb{C}$ admits a $( \epsilon,\eta)$ efficient agnostic learner for any balanced distribution over $\mathfrak{A}$.  
\end{thm}

The result in theorem \ref{thm: al} makes clear that not all sub-population can be efficiently audited for multi-differential fairness. There are many concept classes $\mathbb{C}$ for which agnostic learning is a NP-hard problem, including for any learning methods that outputs a half-space as an hypothesis (see \cite{feldman2012agnostic}). However, there are classes for which efficient agnostic learners exist (see \cite{kearns1994toward}).

\bigskip
Based on theorem \ref{thm: al} and its proof,  we convert the certifying algorithm problem into the following empirical loss minimization: for a sample $\{(x_{i}, a_{i}), y_{i})\}_{i=1}^{m}\sim D^{w}_{f}$, solve

\begin{equation}
\label{eq: risk1}
  opt=\min_{c\in \mathbb{C}}\frac{1}{m}\displaystyle\sum_{i=1}^{m} \mathbbm{1}\left(a_{i}y_{i} = c(x_{i})\right) 
\end{equation}

and then compute $\gamma$ as:
\begin{equation}
\label{eq: risk2}
\gamma = \frac{opt + \hat{\rho} - 1}{4},
\end{equation}
where $\hat{\rho}$ is the sample estimate of $Pr_{w}[A=Y]$.



\subsection{Unbalanced Data}
The risk minimization problem allows to certify efficiently the lack of multi-differential fairness. The catch is that we assume that we have oracle access to importance sampling weights $w$. However, most of the time, importance sampling are unobserved and need to be estimated. Moreover, variance of those estimates are known to be large (see). In this section, we propose a noise infusing technique to certify multi-differential fairness without the need to estimate importance sampling weights.

\paragraph{Importance Sampling and Reweighting}
At issue with unbalanced distribution is that the multi-differential fairness condition in \eqref{eq: mdf_w} includes for $a\in\mathfrak{A}$ the term $w_{S}= Pr[A=a|S]/Pr[A\neq a|S]$:

\begin{equation}
    \label{eq: mdf_nw}
    Pr_{w}[A=a |S, y] \leq \frac{e^{\delta}w_{S}}{e^{\delta} w_{S} + 1},
\end{equation}
Therefore, algorithm \ref{algo: 1}, when applied to unbalanced distribution, cannot distinguish the case of high value of $\delta$ from the case of low value $\delta$ but a high value of $w_{S}$. The latter situation is the result of unbalance in the data that could result from social, cultural or historical biases; the former is an issue with the classifier $f$ itself that needs to be audited for. 

\bigskip
One approach to obtain $w$ is to directly estimate the density $P[A=a|x]$. This idea of importance sampling is used in propensity-score matching methods (see \cite{rosenbaum1983central} and \cite{freedman2008weighting} for this plug-in approach) in the context of counterfactual analysis. However, as noticed in \cite{cortes2010learning}, exact or estimated importance sampling result in large variance in finite sample. In fact, estimating the distribution $P[A=a|x]$ to obtain the weight $w_{a}(x)$ may be an overkill. Instead, we follow \cite{gretton2009covariate} and observe that if $u$ is a weight function, 
\begin{equation}
Pr_{w}[AY = c] \leq Pr_{u}[AY = c] + MMD(uP_{a}, P_{\not a})
\end{equation}
with equality whenever $u=w_{a}$. $MMD$ is the maximum mean discrepancy between the distribution of features $P(x, A=a)$ weighted by $u$ and the distribution of $P(x, A\neq a)$:
\begin{equation}
\label{eq: mmd1}
MMD(uP_{a}, P_{\not a}) = 
\left \Vert E\left[ \frac{1}{n_{a}}\displaystyle\sum_{i, A=a}u(x)\phi(x) -\frac{1}{n_{a^{'}}}\displaystyle\sum_{i, A=a^{'}}\phi(x)\right]\right \Vert^{2}
\end{equation}
where $\phi:\mathfrak{X}\rightarrow \mathfrak{F}$ is a feature map into a feature space $\mathfrak{F}$ ( see \cite{gretton2009covariate}).  In other words, the maximum mean discrepancy between $u P(., A=a)$ and $P(., A\neq a)$ is measured over all function in the reproducing kernel Hilbert space represented by the kernel $k(x,x^{'})=\langle \phi(x)| \phi(x^{'})\rangle$. 

\bigskip
Therefore to audit for multi-differential fairness, our approach consists into finding $c\in \mathbb{C}$, $\phi$ and $u$ that minimizes the empirical counterpart of the upper bound in \eqref{eq: mmd1}:

\begin{equation}
\begin{split}
L(w, c, \phi) = &\frac{1}{N}\displaystyle\sum_{i}u_{i}\mathbbm{1}(c(\phi(x_{i}))\neq a_{i}y_{i}) + Reg(c)  \\
& + \left \Vert \frac{1}{n_{a}}\displaystyle\sum_{i, A=a}u_{i}\phi(x_{i}) -\frac{1}{N -n_{a}}\displaystyle\sum_{i, A\neq a}\phi(x_{i})\right \Vert^{2} + Reg(u)
\end{split}.
\end{equation}
In our implementation $MMD_{NET}$, the feature representation $\phi$ is learned via a neural network that is then shared with both tasks of minimizing the re-weighted certifying risk and the distributional shift between $u P(., A=a)$ and $P(., A\neq a)$. 

\bigskip
Generic uniform convergence argument allows to derive the sample complexity and correctness of our certifying algorithm \ref{algo: 1}. 

\begin{thm}
	\label{thm: corr1}
	(Sample Complexity and Correctness of Algorithm \ref{algo: 1})Let $\epsilon >0$ and $\eta\in(0,1)$. 
	Suppose that $\mathbb{C}$ is a concept class of dimension $d(\mathbb{C})<\infty$. Algorithm \ref{algo: 1} is $(\epsilon, \eta)-$ certifying algorithm for samples of size $m\geq m(\epsilon, \eta, d)$, where
	$$m=$$
\end{thm}



\begin{algorithm}
\caption{Certifying Algorithm }
\label{algo: 1}
\begin{algorithmic}[1]
\State \textbf{Input:}  $\{((x_{i}, a_{i}), y_{i})\}_{i=1}^{m}$, $\mathbb{C}\subset 2^{|\mathfrak{X}|}$, $\lambda_{u}$, $\lambda_{c}$.
 
 \State $\hat{\rho} \gets \frac{1}{m}\displaystyle\sum_{i=1}^{m} \mathbbm{1}(a_{i}=y_{i})$ 
 
 \State $u^{*}, \phi^{*}= argmin \left \Vert \frac{1}{n_{a}}\displaystyle\sum_{i, A=a}u_{i}\phi(x_{i}) -\frac{1}{N -n_{a}}\displaystyle\sum_{i, A\neq a}\phi(x_{i})\right \Vert^{2} + \lambda_{u}\Vert u \Vert ^{2}$
 
 \State $c^{*} = argmin_{c\in \mathbb{C}}\frac{1}{m}\displaystyle\sum_{i=1}^{m} u(x)\mathbbm{1}\left(a_{i}y_{i} = c(\phi(x_{i}))\right) + \lambda_{c}Reg(c)$
 
 \State $\hat{\gamma}a\gets \frac{opt + 1 - \hat{\rho}}{4}$
 
  
\State{\bfseries Return} $\hat{\gamma}-$ unfair
\end{algorithmic}
\end{algorithm}


\subsection{Unfairness Diagnostics: Worst Violation}
Algorithm \ref{algo: 1} presented above allows to certify whether any black box classifier is multi-differential fair with only $O(\log(|\mathbb{C}|)$ samples. However, it does not identify the sub-population in $\mathbb{C}_{\alpha}$ with the strongest violation of multi differential fairness (i.e. with the largest value $\delta$ in \ref{def: mdf}). This is because algorithm \ref{algo: 1} does not distinguish a large sub-population $S$ with low value of $\delta$ from a smaller sub-population with larger value of $\delta$. Finding the strongest violation is useful to (i) diagnostic the source of multi-differential unfairness of a classifier; and, (ii) identify the individuals which could be the most harmed by a classifier's outcome. 

\paragraph{Worst Violation Problem}
The objective is to identify for any $a\in \mathfrak{A}$ the sub-population $S$ in $\mathbb{C}$ that solves:
\begin{equation}
\label{eq: wvio}
    \delta_{m} \equiv \sup_{S\in \mathbb{C}}\log\left(\left.\frac{Pr[A=a|S, y]}{Pr[A\neq a|A, y]}\middle/\frac{Pr[A=a|S]}{Pr[A\neq a|S]}\right)\right. .
\end{equation}
To illustrate the challenges of the worst-violation problem, consider the case of a binary protected attributes $A=\{-1, 1\}$ where there exist $S_{0}, S_{\delta}\in \mathbb{C}$  with no violation of multi differential fairness in sub-population $S_{0}$, but a $\delta-$ multi differential fairness violation in $S_{\delta}$ (e.g. $P[Y| A=1, S_{\delta}] < e^{\delta}P[Y| A=-1, S_{\delta}]$).  Consider $c, c^{'}\in \mathbb{C}$ such that $c(x)=1$ if and only if $x\in S_{\delta}$ and $c^{'}(x)=1$ if and only if $x\in S_{\delta}\cup S_{0}$. Both $c$ and $c^{'}$ have the same accuracy on $S_{\delta}\cup S_{0}$ once the distribution is rebalanced. Therefore, algorithm \ref{algo: 1} will pick indifferently $c$ or $c^{'}$ as unfairness certificate, although $c^{'}$ does not single out $S_{\delta}$ as a worst violation. 

\paragraph{Worst Violation Algorithm}
At issue in the previous example is that for the sub-population $S_{0}$  with no violation of multi differential fairness, choosing $c=1$ or $c=-1$ will lead to same empirical risk used in our certifying procedure \ref{algo: 1}. Our worst violation algorithm \ref{algo: 3} puts a slightly larger weight on samples $((x_{i}, a_{i}), y_{i})$ whenever $a_{i}y_{i}\neq 1$ so that the empirical risk is now smaller if $c=-1$ and there is no violation of multi differential fairness.  More generally, our worst violation algorithm is an iterative procedure such that at each iteration $t$, the weight on samples $((x_{i}, a_{i}), y_{i})$ whenever $a_{i}y_{i}\neq 1$ is increased to $1 + \xi t$, with $\xi > 0$ and the solution $c_{t}^{*}$ of the empirical risk minimization \eqref{eq: risk1} or \eqref{eq: risk2} identifies a sub-population $S_{t}=\{x| c_{t}^{*}(x)=1\}$ for which $\delta \geq \log(1 + \xi t)$. The algorithm \ref{algo: 3} terminates whenever either $|S|\leq \alpha$ or $c^{*}_{t}(x)=-1$ for all $x$. At the last iteration before termination, theorem \ref{thm: algo3_ana} shows that algorithm \ref{algo: 3} will identify a sub-population with a $\delta$-multi differential fairness violation with $\delta$ asymptotically close to $\delta_{m}$. 


\begin{thm}
\label{thm: algo3_ana}
Suppose $\nu > 0, \epsilon >0, \eta\in (0, 1)$ and $\mathbb{C}\subset 2^{\mathfrak{X}}$ is $\alpha-$strong. Denote $\delta_{m}$ the worst violation of multi differential fairness for $\mathbb{C}$ as defined in \eqref{eq: wvio}. With probability $1-\eta$, with $O\left(\right)$ samples and after $O\left(\right)$ iterations, algorithm \ref{algo: 3} learns $c\in \mathbb{C}$ such that 
\begin{equation}
    \log\left(\frac{Pr_{w}[A=a|y, c(x)=1]}{Pr_{w}[A\neq a|y, c(x)=1]} \right) \geq \delta_{m} - \epsilon.
\end{equation}
\end{thm}
\begin{algorithm}
\caption{Worst Violation Algorithm}
\label{algo: 3}
\begin{algorithmic}[1]
\State \textbf{Input:}  $\{((x_{i}, a_{i}), y_{i})\}_{i=1}^{m}$, $\mathbb{C}\subset 2^{|\mathfrak{X}|}$, $\xi$, $alpha$
 
 \State  $\delta_{-1}=0$; $\delta_{0}=1$, $\alpha_{0} =1$ 
 
\While {$\delta_{t} >= \delta_{t-1}$ or $\alpha_{t} > \alpha$}  

 \For  {i=1..m}
 $u_{it} \gets u_{i}(1 + \xi t)$ if $a_{i}y_{i}=-1$
\EndFor
\State $c^{*} = argmin_{c\in \mathbb{C}}\frac{1}{m}\displaystyle\sum_{i=1}^{m} u_{it}(x)\mathbbm{1}\left(a_{i}y_{i} = c(\phi(x_{i}))\right) + \lambda_{c}Reg(c)$

\State $\hat{\delta}_{t}\gets \frac{\displaystyle\sum_{i=1, c(x_{i}=1)}^{m} u_{i}(x_{i})\mathbbm{1}(y_{i}=1)\mathbbm{1}(a_{i}=1)}{ \displaystyle\sum_{i=1, c(x_{i}=1)}^{m} u_{i}(x_{i})\mathbbm{1}(y_{i}=1)\mathbbm{1}(a_{i}=-1)} $

\State $\hat{\alpha_{t}} \gets\frac{\displaystyle\sum_{i=1)}^{m} u_{i}(x_{i})\mathbbm{1}(c_{i}(x_{i})=1)}{ \displaystyle\sum_{i=1}^{m} u_{i}(x_{i})} $
 
\State $t\gets t +1$
 
 \EndWhile   
\State{\bfseries Return} $\ln(\delta_{t})$.
\end{algorithmic}
\end{algorithm}


\subsection{Unfairness Diagnostic: Individual Recourse}
\textbf{[Place holder for now]}. Although multi-differential fairness is only a sub-population level definition of fairness, in this section, we explore how this framework can inform on the harm made by a classifier's outcome on a given individual. At the individual level, the harm is measured by the max-divergence between the distributions $Y|(A=a, x)$ and $Y|(A\neq a, x)$:
\begin{equation}
\delta(x)\equiv max_{y\in Y}\ln\left(\frac{Pr_{w}[Y=y|A=a, x]}{Pr_{w}[Y=y|A\neq a, x]}\right).
\end{equation}

In this section, we propose a lower bound of $\delta(x)$ that can be efficiently computed for any individual $x$. 


\section{Experiments}

\subsection{Synthetic Data}
A synthetic data is constructed by drawing independently two features $X_{1}$ and $X_{2}$ from two normal distribution $N(0, 1)$. We consider a binary protected attribute  $\mathfrak{A}=\{-1, 1\}$ drawn from Bernouilli distribution with $A=1$ obtained with probability $w(x)=\frac{e^{\mu * (x_{1}- x_{2}))^{2}}}{1 + e^{\mu * (x_{1}+ x_{2}))^{2}}}$ and $A=-1$ with probability $1-w(x)$ for $x=(x_{1}, x_{2})$. $\mu$ is an unbalance factor. $\mu=0$ means that the data is perfectly balanced with $w(x)=\frac{1}{2}$. As $\mu$ increases, the distribution $Pr(x|A=1)$ becomes more dense in the for points away from the $45\degree$ diagonal in a $(x_{1}, x_{2})$ plane. The data is labeled according to the sign of $(X_{1} + X_{2} + e) ^{3}$, where is $e$ is a noise drawn from $N(0, 0.2)$. The audited classifier $f$ is a logistic regression classifier that is altered to generate instances of differential unfairness: with probability $1-\nu\in[0, 1)$, the sign of the classifier's outcomes $Y$ from individuals with $A=-1$ is changed from $-1$ to $+1$ if $x^{2}_{1} + x^{2}_{2} \leq 1$; if $A=1$, all classifier's outcomes are changed from $-1$ to $+1$ if $x^{2}_{1} + x^{2}_{2} \leq 1$ . For $\nu=0$, the audited classifier is differentially fair; however, as $\nu$ increases, the half circle $\{(x_{1}, x_{2})|x^{2}_{1} + x^{2}_{2} \leq 1 \mbox{ and } y=-1\}$ there is a fraction $\nu$ of individuals with protected attribute equal to $-1$ who are not treated similarly as individuals with protected value equal to $1$. 

\bigskip
\paragraph{Sample Complexity}
The auditing algorithm $MMD_{NET}$ is trained using a decision tree and a unbalanced data ($\mu=0.2$). Given our setting, for any value of $\nu$ we can compute the actual level of differential unfairness $\gamma_{o}$ and compare it to the value  $\gamma_{a}$ estimated by the auditing algorithm$MMD_{NET}$.  Figure \ref{fig: 1a} plots $\gamma_{a}$ against $\gamma_{o}$ after $100$ simulations for value of $\beta$ varying from $0$ to $0.5$ and fixed sub-population size $\alpha=0.1$. The experiment is conducted with data set of size $n\in \{1000, 5000\}$. The estimated unfairness level $\gamma_{a}$ is unbiased since the plots nearly align with $45\degree$ line. Larger sample ($5K$) have little effect on how bias the auditor $MMD_{NET}$ is, but reduces the variance of the estimated $\gamma_{a}$.  

\begin{figure}[h!]
\begin{subfigure} {.475\linewidth}
\centering
\begin{tikzpicture}
\begin{axis}[
    xlabel={$\gamma_{o}$},
    ylabel={$\gamma_{a}$},
    xmin=0, xmax=0.06,
    ymin=0, ymax=0.06,
    xtick={0, 0.02, 0.04, 0.06, 0.08},
    ytick={0, 0.02, 0.04, 0.06, 0.08},
    legend pos=south east,
    ymajorgrids=true,
    grid style=dashed,
    width=\linewidth
]
\addplot[mark=none] table [x=gamma, y=gamma, col sep=comma]
     {../results/synth_exp_sample_size_var.csv};
 \foreach\g in{1000, 5000}{
                \addplot  
                plot [error bars/.cd, y dir = both, y explicit]
                 table [discard if not={size}{\g}, x=gamma, y=estimated_gamma, y error=gamma_deviation, col sep=comma]
     {../results/synth_exp_sample_size_var.csv};
            }
            \legend{$45\degree$ line, $n=1K$, $n=5K$}

\end{axis}
\end{tikzpicture}
\caption{Effect of unfairness intensity $\beta$ on auditor's performance.}
\label{fig: 1a}
\end{subfigure}
 \hfill%
\begin{subfigure} {.475\linewidth}
\centering
\begin{tikzpicture}
\begin{axis}[
    xlabel={$\mu$},
    ylabel={$\gamma_{a}-\gamma_{o}$},
    xmin=-0.30, xmax=0.3,
    ymin=-0.01, ymax=0.05,
    xtick={-0.3, -0.2, -0.1, 0.0, 0.1, 0.2, 0.3},
    ytick={-0.01, 0, 0.01, 0.02, 0.03, 0.04, 0.05},
    legend pos=north west,
    ymajorgrids=true,
    grid style=dashed,
    width=\linewidth
]
 \foreach\g in{Uniform, IS, MMD_NET}{
                \addplot  
                 table [discard if not={balancing}{\g}, x=unbalance, y=bias , col sep=comma]
     {../results/synth_exp_unbalance_3a.csv};
            }
            \legend{$UW$, $ISW$, $MMD_{NET}$}

\end{axis}
\end{tikzpicture}
\caption{Effect of unbalanced data on auditor's performance.}
\label{fig: 1b}
\end{subfigure}
\caption{Certifying $\gamma$- multi differential unfairness. The auditor is a decision tree whose depth is tuned using a $5$-fold cross validation.  The weights function $u$ is obtained using a neural network with four hidden layers with eight neurons each. The plots show the average of $100$ experiments. Figure \ref{fig: 1a} auditor's bias when estimating $\gamma_{a}$ is measured by deviations from the $45\degree$ line; the unbalance parameter $\mu$ is set to $0$; the size $\alpha$ of sub-population with a fairness violation is set to $0.1$; and, the intensity $\beta$ of the fairness violation varies from $0$ to $0.5$. Figure \ref{fig: 1b}: bias is measured as the difference $\gamma_{a}-\gamma_{o}$; the size $\alpha$ of sub-population with a fairness violation is set to $0.1$; the intensity $\beta$ of the fairness violation is set to $0.5$; the balancing parameter $\mu$ varies from $-0.3$ to $0.3$.  } 
\end{figure}

\paragraph{Unbalanced Data}
To test the performance of our certifying algorithm on unbalanced data, we repeat the previous experiment with $\mu$ varying from $-0.3$ to $0.3$. We compare the performance of three reweighting approaches: uniform weight ($UW$); direct use of the importance sampling weights $w$ ($ISW$); . Figure \ref{fig: 1b} shows that absent of a reweighting scheme ($UW$) the certifying algorithm fails to estimate correctly the classifier's unfairness if $\mu \neq 0$. This is in line with the observation made in section 3 that the auditing algorithm needs to control for the information leaked by the auditing features $x$ when measuring the additional leakage from the classifier's outcome $y$. Our preferred method $MMD_{NET}$ performs well as there is little bias in the estimate $\gamma_{a}$ even at large value of the unbalancing factor $\nu$. Note using importance sampling weights directly does worse than a uniform sampling: this confirms previous observations in the literature that in finite sample, the variance of the importance sample weights can affect be detrimental to a re-balancing approach. 

\paragraph{Concept Class}
Figure \ref{fig: 2a} runs a similar experiment but varies with decision trees of depth varying from $5$ to $40$. At small level of unfairness, a less complex concept class $\mathbb{C}$ generates less bias in the estimate of $\gamma$. A shorter decision tree out-performs significantly deeper structures. This result, in line with the sample complexity presented in theorem \ref{thm: corr1} justifies the concept of multi fairness: in order to be statistically meaningful, the granularity of the sub-populations for which multi differential fairness is audited for is bounded by the complexity of $\mathbbm{C}$. 


\begin{figure}[h!]
	\begin{subfigure} {.475\linewidth}
		\centering
		\begin{tikzpicture}
		\begin{axis}[
		xlabel={$\gamma_{o}$},
		ylabel={$\gamma_{a}$},
		xmin=0, xmax=0.06,
		ymin=0, ymax=0.06,
		xtick={0, 0.02, 0.04, 0.06, 0.08, 0.1},
		ytick={0, 0.02, 0.04, 0.06, 0.08, 0.1},
		legend pos=north west,
		ymajorgrids=true,
		grid style=dashed,
		width=\linewidth
		]
		\addplot[mark=none] table [x=gamma, y=gamma, col sep=comma]
		{../results/synth_exp_sample_size_var.csv};
		\foreach\g in{dt, rf, svm_linear, svm_rbf}{
			\addplot  
			table [discard if not={auditor}{\g}, x=gamma, y=estimated_gamma , col sep=comma]
			{../results/synth_exp_auditor_2a.csv};
		}
		\legend{$45\degree$ line, Tree, Random Forest, SVM RBF, SVM Linear}
		
		\end{axis}
		\end{tikzpicture}
	\end{subfigure}
	\caption{Effect of auditor's class on auditor's performance.}
	\label{fig: 2a}
	\caption{Certifying $\gamma$- multi differential unfairness. The auditor is a decision tree whose depth is tuned using a $5$-fold cross validation.  The weights function $u$ is obtained using a neural network with four hidden layers with eight neurons each. The plots show the average of $100$ experiments. Figure \ref{fig: 2a} auditor's bias when estimating $\gamma_{a}$ is measured by deviations from the $45\degree$ line; the unbalance parameter $\mu$ is set to $0.2$; the size $\alpha$ of sub-population with a fairness violation is set to $0.1$; and, the intensity $\beta$ of the fairness violation varies from $0$ to $0.5$.  }  
\end{figure}

\paragraph{Worst Violations}
As noted in section 3, the certifying algorithm does not allow to distinguish between a very intense (high $\delta$)
on a small sub-population (small $\alpha$) from a less intense violation (small $\delta$) on a large sub-population (large $\alpha$). We test whether algroithm \ref{algo: 3} is able to identify that in the synthetic data, the differential fairness violation occurs in the sub-space $\{(x_{1}, x_{2})|x^{2}_{1} + x^{2}_{2} \leq 1 \mbox{ and } y=-1\}$. Figure \ref{fig: 3a} shows that there is little bias in the estimate of $\delta_{a}$ for intermediate values of $\delta$. The algorithm underestimates the true value of $\delta_{o}$ for large violations of differential fairness. 
\begin{figure}[h!]
	\begin{subfigure} {.475\linewidth}
		\centering
		\begin{tikzpicture}
		\begin{axis}[
		xlabel={$\delta_{o}$},
		ylabel={$\delta_{a}$},
		xmin=0, xmax=3,
		ymin=0, ymax=3,
		xtick={0, 0.5, 1, 1.5, 2.0, 2.5, 3.0},
		ytick={0, 0.5, 1, 1.5, 2.0, 2.5, 3.0},
		legend pos=north west,
		ymajorgrids=true,
		grid style=dashed,
		width=\linewidth
		]
		\addplot[mark=none] table [x=delta, y=delta, col sep=comma]
		{../results/synth_exp_violation_delta.csv};
		\foreach\g in{0.015}{
			\addplot  
			table [discard if not={step}{\g}, x=delta, y=estimated_delta , col sep=comma]
			{../results/synth_exp_violation_delta.csv};
		}
		\legend{$45\degree$ line, $\xi=0.01$}
		
		\end{axis}
		\end{tikzpicture}
	\end{subfigure}
	\label{fig: 3a}
	\caption{Finding $\delta$ for the worst violation of multi-differential fairness. The auditor is a decision tree whose depth is tuned using a $5$-fold cross validation.  The weights function $u$ is obtained using a neural network with four hidden layers with eight neurons each. The plots show the average of $100$ experiments. auditor's bias when estimating $\gamma_{a}$ is measured by deviations from the $45\degree$ line; the unbalance parameter $\mu$ is set to $0.2$; the size $\alpha$ of sub-population with a fairness violation is set to $0.1$; and, the intensity $\beta$ of the fairness violation is set to $0.5$.  The bias in the estimated $\delta$ is measured as the difference $\delta_{a}-\delta_{o}$  }  
\end{figure}


\subsection{Case Study: COMPAS}
We chose to apply first our method to the COMPAS algorithm because it is a
 widely used to assess the likelihood of a defendant to become a recidivist. Records from Broward County, Florida are publicly available and expansive analysis of the COMPAS fairness have been already carried out. Our novelty is to apply our multi-differential fairness framework and explore whether there exists in those records group of individuals that could argue for a disparate treatment akin to an "intentional" discrimination. 

\paragraph{Data Description}
The data collected by ProPublica in Broward County from 2013 to 2015  contain $7K$ individuals along with a risk score and a risk category assigned by COMPAS. We transform the risk category into a binary variable equal to $1$ for individuals assigned in the high risk category (risk score between $8$ and $10$). The data provides us with information related to the historical criminal history, misdemeanors, gender, age and race of each individual.   

 \paragraph{Certifying the Lack of Differential Fairness}
First, we assess whether the COMPAS risk classification is multi-differential fair. The data is randomly split into train and test sets using $70\%/30\%$ ratio. The auditor uses a four layers fully connected neural network to re-balance the data and a random forest to certify differential fairness. The assessment is made for a binary protected attribute:  whether an individual self-identified as Afro-American.  We find a significant level of differential unfairness in the COMPAS risk classification (see Table \ref{tab: 1}) when the binary protected attribute is whether an individual self-identified as Afro-American. The unfairness is slightly stronger with a auditing feature space limited to the count of prior felonies and the degree of the current charge.On the other hand, we do not find any evidence that the COMPAS risk classification violates differential fairness when the protected attributes is a binary variable indicating whether an individual self-identifies a Male.

\begin{table}[h!]
	\begin{tabular}{lll}
		Features & Race    & Gender  \\
		\hline
		I, II, III, IV, V         & 0.023761 $\pm 0.004312$ &   $-0.0001 \pm 0.0006$        \\
		I, II, III         & 0.023599 $\pm 0.00384$&     $-0.000037 \pm 0.0002$       \\
		I, II          & $0.027382 \pm 0.019383$ & $-0.00005 \pm 0.0001$         \\
		\hline 
	\end{tabular}
	\label{tab: 1}
	\caption{Certifying the lack of differential fairness in COMPAS risk classification. Features are as follows: I: count of prior felonies; II: degree of current charge (criminal vs non-criminal); III: age; IV: count of juvenile prior felonies; V: count of juvenile prior misdemeanors. Standard deviations are obtained by drawing $100$ train/test splits. In the first column, Race, the protected attribute is a binary variable indicating whether an individual is self-identified as Afro-American; in the second column, Gender, the protected attribute is a binary variable indicating whether an individual is self-identified as Male}
\end{table}

\paragraph{Worst Violations}
The results in Table \ref{tab: 1} do not allow to distinguish the case of a small violation of differential fairness spread evenly across the whole feature distribution from the case of large violations concentrated on small sub-populations. 


\paragraph{Disparate Treatment}
Pick up Alice, Bob, Carol and Dick who belongs to sub-population for which a disparate treatment can be shown. 


\subsection{Other Datasets: Group Fairness vs. Multi-Differential Fairness}

\bibliographystyle{plain} % We choose the "plain" reference style
\bibliography{references}

\end{document}